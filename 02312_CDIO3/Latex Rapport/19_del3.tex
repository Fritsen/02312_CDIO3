\documentclass{article}
\usepackage[T1]{fontenc} %Better support for european accented chars
\usepackage[utf8]{inputenc} %Enables danish chars and other good stuff
\usepackage{lmodern} 
\usepackage{amssymb}
\usepackage[danish]{babel}
\usepackage{pstricks-add}
\usepackage{lastpage} %Package for getting the number of the last page.
\usepackage{graphicx} %Allows insertion of graphics
\usepackage{amsmath} %Gives better math support.
\usepackage{amsfonts} %Gives additional fonts.
\usepackage{amssymb} %Adds additional symbols.

%Pakker der bruges for at lave klikbare links i pdf filer
\usepackage{varioref} %Gives the \vref{lbl} command that make nice references that includes page numbers.
\usepackage[pdftex, colorlinks=false, pdfauthor={\authorText}, pdftitle={\titleText}]{hyperref} %Makes references clickables in the pdf file. colorlinks can be set to true for colored links.
\usepackage{memhfixc} %Solves problems with hyperref in Memoir, to be loaded after hyperref.

\usepackage{url}
\usepackage{pdfpages}
\usepackage{listings}
\usepackage{color}
\usepackage{todonotes}
\usepackage{url} % Nice url look
\usepackage[backend=biber,style=alphabetic]{biblatex} %Bibliography package.
%\usepackage{anysize}
%\marginsize{3cm}{3cm}{1.5cm}{4cm} % venstre, højre, top, bund.
\lstset{literate=%
{æ}{{\ae}}1
{å}{{\aa}}1
{ø}{{\o}}1
{Æ}{{\AE}}1
{Å}{{\AA}}1
{Ø}{{\O}}1
}

\definecolor{mygreen}{rgb}{0,0.6,0}
\definecolor{mygray}{rgb}{0.5,0.5,0.5}
\definecolor{mymauve}{rgb}{0.5,0,0.3}
\definecolor{myblue}{rgb}{0,0.4,0.9}
\pagestyle{plain}
\begin{document}
\section*{Projektopgave efterår 2013 - jan 2014}
\section*{02312-14 Indledende programmering og 02313 Udviklingsmetoder til IT-Systemer}
Projektnavn: \textcolor{red}{del2}
Gruppe nr: \textcolor{red}{19}
Afleveringsfrist: \textcolor{red}{mandag den 11/11 2013 Kl. 5:00}
\\\\
Denne rapport er afleveret via Campusnet (der skrives ikke under)
\\\\
Denne rapport indeholder \textcolor{red}{57} sider incl. denne side
\\\\
Studie nr, Efternavn, Fornavne
\\\\
\textcolor{red}{s110795, Mortensen, Thomas Martin}
\\
Kontakt person (Projektleder)
\\\\
\includegraphics[scale=0.5]{ThomasM.jpg}
\\\\
\textcolor{red}{s113577, Johansen, Chris Dons}
\\\\
\includegraphics[scale=0.5]{ChrisJ.jpg}
\\\\
\textcolor{red}{s123897, Ahlgreen, Thomas Kamper}
\\\\
\includegraphics[scale=0.5]{ThomasA.jpg}
\newpage
\section*{Timeregnskab}
\begin{tabular}{|c|c|c|c|c|c|c|r|} \hline
 Dato & Deltager & Design & Implementering & Test & Dok. & Andet & I alt \\ \hline
&&&&&&& \\ \hline
 18/10-13 & Thomas Mortensen & 2& & & & & 2\\ \hline
 18/10-13 & Chris Johansen & 2& & & & &2 \\ \hline
 18/10-13 & Thomas Ahlgreen & 2& & & & & 2\\ \hline
 &&&&&&& \\ \hline
 25/10-13 & Thomas Mortensen & & 2,5& & & &2,5 \\ \hline
 25/10-13 & Chris Johansen & &2,5 & & & &2,5 \\ \hline
 25/10-13 & Thomas Ahlgren & & 2,5& & & &2,5 \\ \hline
 &&&&&&& \\ \hline
 1/11-13 & Thomas Mortensen & &2,5 & & & &2,5 \\ \hline
 1/11-13 & Chris Johansen & &2,5 & & & & 2,5 \\ \hline
 1/11-13 & Thomas Ahlgren & &2,5 & & & & 2,5 \\ \hline
 &&&&&&& \\ \hline
 8/11-13 & Thomas Mortensen & & &1 &6 & &7 \\ \hline
 8/11-13 & Chris Johansen & & &1 &6 & &7 \\ \hline
 8/11-13 & Thomas Ahlgren & & &1 &6 & &7 \\ \hline
 &&&&&&& \\ \hline
 diverse & Thomas Mortensen & & & &4 & & \\ \hline
 diverse & Chris Johansen & & & & &2 & \\ \hline
 diverse & Thomas Ahlgren & & & & & & \\ \hline
 &&&&&&& 48\\ \hline
\end{tabular}
\newpage
\tableofcontents
\newpage
\section{Indledning}
Spilfirmaet IOOuterActive har givet os endnu en opgave.

Kundens vision er en udvidelse af vores allerede udviklede programmer fra \cite{del1} og \cite{del2}.
Denne gang ønsker kunden at lave flere forskellige typer felter. Disse felter skal implementeres på en "rigtig" spilleplade, hvor man kan gå i ring. Samtidig ønskes der mulighed for 2-6 spillere. Spillerens konto skal sættes med et fast beløb fra start, og spillet skal slutte, når en spiller er gået bankerot.


Projektet forventes at have elementer fra både \textbf{FURPS+} og \textbf{GRASP}. Herudover er der lavet specifikke krav til hvilke artefakter, der skal ingå i krav, analyse, kode, test, og designdokumentation.



OBS: alle gruppemedlemmer er lige ansvarlige for alle dele af vores dokumentation
\section{Kravspecificering og Use cases}
I dette afsnit vil vi beskrive vores kravspecificering og vores Use cases, som vi har udarbejdet til denne rapport.
\subsection{Kravspecificering}
\todo[inline]{Thomas M: Indsætte de få tvivlskrav, der er i dropbox}
I dette afsnit har vi taget udgangspunkt i kundens vision. Vi har læst den igennen, og stillet spørgsmålstegn, ved de ting, der kunne være tvivl om. Kunden er dog blevet til kravspecificering i forløbet, og derfor var der få tvivlsspørgsmål.
\begin{enumerate}
\item Hvad skal der ske, hvis en spiller ikke kan betale det han skal?
\item Skal man betale, hvis man lander på et felt, der er ejet af en bankerot spiller?
\item Skal man have mulighed for at købe eet felt, der tidliger var ejet af en spiller, der nu er gået bankerot?
\end{enumerate}
Vores kontakt i firmaet, som er vores projektleder, har besvaret disse punkter på følgende måde
\begin{enumerate}
\item Spilleren der ikke kan betale, må betale det han har, og går herefter bankerot.
\item Feltet er ikke længere ejet af nogen, og derfor skal der heller ikke betales.
\item Da feltet ikke længere ejes, skal man kunne købe det.
\end{enumerate}
\subsection{Use Cases}
\todo[inline]{Thomas A: Indsæt tekst fra fully dresses og brief + diagram}
Vi vil i dette afsnit beskæftige os med 2 use cases, som er beskrevet i nedenstående diagram. Til vores spil har vi lavet en fully dressed use case, og til vores test har vi blot lavet en brief use case.
\subsubsection{Use Case Diagram}
\begin{figure}[ht]
\centering
\includegraphics[scale=0.5]{UseCaseDiagram.jpg}
\caption[<Text for the list of figures>]{Use Case Diagram}
\label{fig:figure 2}
\end{figure}
\newpage
\subsubsection{Fully Dressed Use Case}
Vi har i dette afsnit beskrevet vores use case med en fully dressed use case for spil.
\subsubsection*{Use Case}
Spil
\subsubsection*{Scope}
Terningspil
\subsubsection*{Level}
User Goal
\subsubsection*{Primary actor}
Spiller
\subsubsection*{Stakeholders and Interests}
Spiller: vil vinde spillet
\subsubsection*{Preconditions}
Eclipse er installeret på maskinen.
\\
Programmet er startet.
\subsubsection*{Main Success Scenarie:}
\begin{enumerate}
\item Spiller starter spillet
\item Spiller indtaster Navn
\item Spiller Ruller med terningerne
\item Spiller lander på et felt med positiv konsekvens
\item Spiller modtager point
-- Spiller gentager punkt 3-5 indtil 3000 eller mere er opnået
\item System checker vinder
\item System viser vinder
\item System lukker ned
\end{enumerate}
\subsubsection*{Extensions Alternative scenarier:}
*a - til hvert et tidspunkt
\begin{enumerate}
\item Spiller lukker spillet ved at taste "q"
\end{enumerate}
4a. Spiller lander på et felt med negativ konsvens
\begin{enumerate}
\item Spiller får trukket point
\item Spiller ryger under 0 point
-- punkt 6-8 i main success aktiveres
\end{enumerate}
4b. Spiller lander på et felt uden konsekvens
\begin{enumerate}
\item Spiller for trukket point
\item a Spiller ryger under 0 point
-- punkt 6-8 i main success aktiveres
\item b Spiller rammer felt 10. uden at komme under 0, hviket giver ekstra tur. 
-- punkt 3 i main success aktiveres
\end{enumerate}
\subsubsection*{Specielle Krav:}
\begin{itemize}
\item Skal kunne køre på en Windows maskine med Java EE på DTU's computere
\item Skal kunne spilles af en almindelig bruger
\end{itemize}
\subsubsection{Brief Use Case Account test}
Vi har valgt at lave en brief use case over vores test.
\\

\textbf{Account Test}
En udvikler tester klassen med forskellige grænse og middelværdier, der er foruddefineret i klassen. Resultaterne printes ud i konsollen, og udvikleren ser om resultaterne fra testen, stemmer overens med de forventede resultater.
\section{Domænemodel}
\todo[inline]{Thomas M: indsæt diagram og forklar}
Da vi har kunnet bruge en stor del af vores domænemodel fra sidst, har vi valgt ikke at lave en navneordsanalyse, da vi ville bruge for lang tid på at lave den, i forhold til udbyttet af denne. Dette er årsagen til, at vi starter med domænemodellen.
\\

Vores domæne model er lavet før begyndelse af implementering. Det er udført forinden for at få et overblik over hvilke elementer, vi kunne have behov for i vores program. 
\\


Vi har i \vref{fig:domain} vist vores hoved domæner. \textit{Game}, som er selve spillet. \textit{Gameboard}, der holder styr på indholdet i spillet. \textit{Field}, som indeholder en score for feltet, og om det giver ekstra tur. \textit{Player}, der bruger spillet. \textit{Account} holder styr på point. \textit{DieCup}, som skal rulle vores terninger. Til sidst har vi \textit{Die} som indeholder terningeværdier, vi skal bruge til at udregne, hvor spilleren lander.

\begin{figure}[!ht]
    \centering
    \includegraphics[width=1\textwidth]{Domainmodel.pdf}
    \caption[<Text for the list of figures>]{Domæne Model}
    \label{fig:domain}
\end{figure}
\section{BCE model}
\todo[inline]{Thomas M: indsæt model og forklar}
\begin{figure}[ht]
\centering
\includegraphics[width=1\textwidth]{BCEModel.pdf}
\caption[<Text for the list of figures>]{BCE Model}
\label{fig:bcemodel}
\end{figure}
I denne CDIO opgave, har vi igen benyttet os af en BCE model til at skabe oveblik over vores kendskab i koden.
\\
Vi har denne gang ændret 
\\
Det vil sige at vi i denne model kun har tilføjet vores nye entitetsklasser. Det handler om entiteterne \textit{Account}, \textit{Field}, og \textit{GameBoard}. Vi har også lavet en ny boundary vi kalder \textit{Graphic}, som bearbejder beskeder til vores importerede \textit{GUI}.
\\
Som det fremgår af diagrammet, styrer controlleren \textit{Game} stadig spillet. Det vil sige at vores controller tildeler ansvaret ud til de nære klasser.
\\
\textit{Account} er en pengebeholdning der styres af \textit{Player} klassen. Vores controller har derfor kun brug for direkte kendskab til \textit{Player}.
\\\\
\textit{Field} holder styr på felterne i spillet, og vores \textit{Gameboard} holder styr på felterne. Vores \textit{Game} controller har dermed kendskab til felterne i spillet gennem vores \textit{Gameboard}.
\\


Ud fra disse informationer kan controlleren så sende besked til vores boundary's \textit{TUI} og \textit{Graphic}, som viser brugeren konsekvenserne i spillet grafisk og i tekst. Dette er hvad \vref{fig:bcemodel} beskriver
\section{Systemsekvens Diagram}
\todo[inline]{Thomas A}
\begin{figure}[!ht]
\centering
\includegraphics[scale=0.4]{SystemSequenceDiagramDieGame.jpg}
\caption[<Text for the list of figures>]{Systemsekvens Diagram}
\label{fig:figure 2} 
\end{figure}
Vores System sekvens diagram tager udgangspunkt i vores fully dressed use case: spil. Det første der sker, er at selve spillet startes. Herefter finder vi ud af hvor mange spillere, der er. Spillernes startbalance bliver nu sat. Herefter vil spillet kører i en løkke, indtil en af spillerne har tabt eller vundet. Løkken printer turen på spilleren. Terningen kastes. Spilleren får konsekvensen af det felt han lander på. Beløbet bliver overført til spillerens konto. Spilleren får nu vist hvad balancen er på kontoen. Der bliver kontrolleret, hvor vidt spilleren har opnået point nok til at vinde eller har mistet alle sine penge (Her mangler en metode til at kontrollere balancen på spillerens konto). Hvis det er tilfældet hopper vi ud af løkken. Hvis det er et bestemt felt på pladen, skal spillet give spilleren en ekstratur (Pilen vender den forkerte vej i diagrammet). Dette vil give ham en tur mere i løkken. Hvis ingen af disse ting forekommer, giver spillet turen videre. Hvis en spiller har vundet, skriver konsollen, hvilken spiller det er og hans point. Hvis en spiller har tabt, skriver konsollen, hvilken spiller der så har vundet og hans point. Efter det lukker spillet. Spilleren har hele tiden mulighed for at stoppe spillet med tasten "q".
\\
Systemsekvensdiagrammet er vigtigt at have med, da det vise hvordan systemet reagere på udefrakommende inputs. Det giver også indblik i hvordan systemet skal fungere uden at skulle skrive en masse kode.
\section{Kode}
\todo[inline]{Chris: skriv implementering af kode.}
\subsection{Struktur og pakker}
Ligesom i CDIO del1 projektet, er programmet skrevet med fokus på at overholde BCE-modellen så meget som overhovedet muligt. Derfor er programmet også i dette projekt opdelt i pakker med navne svarende til BCE modellen, dvs. TUI og Graphic i ”Boundary”, Game og Main i ”Controller” og Account, Die, DieCup, Field, GameBoard og Player i ”Entity”. Med andre ord er klasserne opdelt i pakker efter hvilken type de er.
\\

Undtagelsen her er de forskellige test-klasser, som er lagt ned i en pakke, ”TestTools”, for sig selv. Dette er gjort ud fra en ide om at det vil være udviklere der benytter dem, og at de typisk nemmest vil kunne benyttes ved at omdøbe og indsætte dem direkte på andre klassers pladser. Således er der ikke nogen direkte henvisninger til dem i koden til det egentlige program, som almindelige brugere vil opleve.
\\


Herunder en gennemgang af de forskellige klasser og deres funktion.
\\
\subsection{TUI}
Denne klasse håndterer alt hvad der kommer ind og ud af konsollen. For en grundigere beskrivelse af denne klasses funktionalitet henvises til det tilsvarende afsnit i CDIO del1 rapporten (7.0.4, side 12).
Nogle metoder har andre navne og udskriver selvfølelig noget andet tekst, men den overordnede virkemåde er fuldstændigt identisk.
\subsection{Graphic}
Denne klasse styrer alt hvad der skal ændres på den grafiske brugerflade. Der kan argumenteres for, at denne klasse burde navngives ”GUI” for at overholde mønsteret fra tekst brugerfladen, men dette ville besværliggøre udviklingen, fordi det bibliotek der bliver stillet til rådighed til brugerfladen også bruger denne navngivning. Således vælger vi blot at kalde klassen ”Graphic”, med henvisning til at den styrer den grafiske del af programmet.
\\
Nogle metoder kalder blot de tilsvarende metoder i biblioteket direkte (setDice, addPlayer, close), men vi vælger at kalde dem gennem Graphic for at have henvisningerne til GUI-biblioteket samlet det samme sted. Denne fremgangsmåde gør det langt nemmere at vedligeholde programmet i tilfælde af ændringer/opdateringer af det bibliotek der benyttes.
\\
Graphic indeholder desuden en ”moveCar”-metode, som blot kalder to metoder i GUI-biblioteket – en til at fjerne brugerens bil(er) fra spillepladen, og en til at sætte bilen på et nyt felt.
\\


Den interessante del af Graphic-klassen er imidlertid ”setupFields”-metoden. Her udføres en række kald til hjælpemetoden ”createField”, der igen kalder metoder i GUI-biblioteket til at ændre titel, undertitel og beskrivelse af et felt.
De felter, som benyttes til dette spil (2 til 12) opsættes med en liste af kald, som indeholder informationer om navn og score for hvert felt. Der kan argumenteres for, at opsætningen af felter på GUI’en kunne genbruge navne fra TUI’en og felt scoreværdier fra GameBoard, men vi vælger undlade at gøre dette, simpelthen for at holde koblingen så lav som muligt, og for at gøre programmet mere fleksibelt ift. ændringer.
F.eks. kunne man måske forestille sig at en kunde kun ville betale for at få oversat enten TUI eller GUI i programmet, hvilket ikke ville være muligt hvis de hænger sammen, eller måske mere sandsynligt, at kunden var interesseret i at modificere programmet til kun at have en GUI, og derved fjerne TUI-klasse. Uanset hvad vil det i sådanne tilfælde være en fordel, at de forskellige klasser ikke er koblet for meget sammen, og det vurderer vi er vigtigere, end at spare et par linjers kode ved at kunne nøjes med at skrive felternes navne et enkelt sted.
\subsection{Account}
Denne klasse repræsenterer en konto, med en mængde penge/score. Klassen er ganske simpel, i den forstand at den kun indeholder en enkelt attribut med tilhørende get og set metoder, samt en metode der kan tilføje til den eksisterende score.
\\

Det specielle ved Account-klassen er, at den sikrer at en Account-balance ikke kan blive negativ, og giver en tilbagemelding på om en transaktion er gennemført eller ej.
\\
Klassen indeholder både en konstruktør som tager en ”oprettelses-balance” (initialAccountValue), samt en konstruktør som ikke tager nogen argumenter, og så blot opretter kontoen med en værdi på 0.
\\
\subsection{Die}
Denne klasse er identisk med den Die-klasse der blev benyttet i CDIO del1. For en beskrivelse af klassen henvises til det tilsvarende afsnit i CDIO del1 rapporten (7.0.6, side 13).
\subsection{DieCup:}
Denne klasse læner sig kraftigt op ad den DieCup-klasse der blev benyttet i CDIO del1, der er blot fjernet noget overflødig funktionalitet i den udgave der er benyttet i dette projekt. For en beskrivelse af funktionaliteten i denne klasse henvises derfor til det tilsvarende afsnit i CDIO del1 rapporten (7.0.7, side 13).
\subsection{Field}
Denne klasse er skrevet til at bære de relevante informationer om et felt på spillepladen. Navn og beskrivelse er lagt i brugerfladeklasserne, for at gøre det nemmere at oversætte, så det eneste Field-klassen skal indeholde, er den effekt feltet skal have på en spiller – dvs. hvilken score feltet giver, og om feltet skal give en ekstra tur.
\\

Hvad et felt skal gøre, kan ikke ændre sig i løbet af et spil, så der er ingen set-metoder til attributterne – de sættes med konstruktøren, og ændrer sig ikke efterfølgende. Dvs. klassen har en konstruktør som tager et heltal for hvilken score feltet giver, samt en boolsk værdi for om feltet giver en ekstra tur.
De fleste felter giver ikke en ekstra tur, så klassen indeholder også en konstruktør som kun tager et heltal for score, og så blot sætter værdien for ekstra tur til falsk. På den måde kan der spares lidt kode ved oprettelsen af alle felterne.
\subsection{GameBoard}
Denne klasse repræsenterer en spilleplade i spillet. Den indeholder blot et array med alle de felter som skal bruges, samt en get-metode til at få fat i et felt på listen. Felterne oprettes og tildeles deres værdier i konstruktøren, og ændres ikke efterfølgende.
\\

Der kan argumenteres for, at listen med felter fint kunne ligge i Game-controlleren, men for at få højst mulig sporbarhed ift. et ”rigtigt” spil, hvor der naturligvis vil være en spilleplade, vælger vi at lave en klasse til at indeholde felterne. Som en bonus fjerner denne løsning også noget kompleksitet fra controlleren i forhold til oprettelse af felterne.
\subsection{Player}
Denne klasse indeholder data om en spiller. I dette spil er der kun behov for at lagre en score og et navn, men ved at benytte en struktur med en decideret klasse til at indeholde spillerdata, er programmet forberedt til at gemme andre informationer om en spiller – f.eks. spillerens position på spillepladen, eller hvad der blev slået sidst osv..
\\

Scoren for en spiller gemmes i dette program i en klasse til formålet – Account. Når et nyt objekt af Player-klassen oprettes, laves der således også blot en ny Account, som så holder styr på spillerens score. Se evt. beskrivelse af Account-klassen tidligere i denne rapport.
\subsection{Main}
Main-klassen benyttes kun til at oprette og starte Game-kontrolleren, og indeholder således hverken logik, metoder eller data.
\subsection{Game}
Dette er klassen som indeholder selve logikken i spillet, og her vil derfor være en del ting der er interessante at nævne omkring beslutninger ift. implementeringen.
\\

Først og fremmest er det værd at bemærke, at spillerne i programmet lagres i et array af spillere. Det betyder dels at det bliver nemmere at lave en generisk løsning på logikken for en spillers tur, men dels også at det er nemt at udvide spillet til flere spillere. Faktisk er antallet af spillere angivet som en konstant i begyndelsen af klassen, og hele programmet er skrevet med tanke på, at det skal kunne virke med mere end 2 spillere – på den måde er programmet godt forberedt til udvidelse.
\\


Et eksempel på dette er oprettelsen af nye player-objekter, som bliver kørt i Game-konstruktøren.
\begin{figure}[!ht]
\centering
\includegraphics[scale=0.4]{Game-illustration1.jpg}
\caption[<Text for the list of figures>]{Spillere oprettes i en løkke}
\label{fig:figure 2} 
\end{figure}
Objekterne oprettes i en løkke, som stoppes når antallet af oprettede spillere nå op på det antal, som er angivet i konstanten i starten af programmet.
\\

Foruden player-objekterne oprettes i konstruktøren også diverse andre objekter af klasser, som bruges i kontrolleren – dieCup, GameBoard samt en scanner. Desuden åbnes GUI’en fra konstruktøren, og felterne på GUI’en indstilles til noget der er passende for spillet.
\\

Den øvrige logik ligger i ”startGame”-metoden, som også kaldes direkte fra main.
TUI’en kaldes for at udskrive spillets regler, og herefter bedes alle spillere indtaste deres navn. Igen er der tale om en fleksibel løsning, som vil virke med flere end 2 spillere.
\begin{figure}[!ht]
\centering
\includegraphics[scale=0.4]{Game-illustration2.jpg}
\caption[<Text for the list of figures>]{Indlæs navne i en løkke}
\label{fig:figure 2} 
\end{figure}
Igen foregår det selvfølgelig i en løkke, og igen kører løkken bare indtil der er kørt lige så mange gange som der er angivet, at der skulle være spillere.
Inde i løkken udskrives så først en lille tekst, som beder brugens indtaste et navn, derefter indsamles det indtastede fra konsollen, hvorefter det gemmes i det player-objekt som passer til spilleren, og til sidste opdateres GUI’en med den nye information.
\\

Herefter starter selve logikken for spillernes ”ture” i spillet. Turene køres i en endeløs løkke (”while(true)”), som så kan afbrydes ved forskellige scenarier. Denne løsning er valgt, dels fordi der er mange forskellige ting der kan afbryde spillet (en spiller trykker ”q”, en spiller vinder, en spiller taber), dels fordi vi er interesseret i at kunne afbryde spillet midt i en tur. F.eks. giver det ikke meget mening at der køres en hel tur færdig med score osv., hvis en spiller har trykket ”q” for at afslutte spillet, og det giver heller ikke meget mening at udskrive endnu en status i spillet, og tjekke for om en af spillerne har vundet, hvis der allerede er en spiller som har tabt osv..
\\

Selve tur-logikken starter med at udskrive navnet på den spiller, som skal slå. Herefter ventes på input fra konsollen. Når der er givet et input, tjekkes om inputtet er ”q” – er det det, afsluttes spillet. Ellers fortsættes, og der slås med terningerne. Ud fra værdien af terningerne, findes så det felt som spilleren er landet på, og scoren for det pågældende felt hentes. Denne score tilføjes så spillerens player-objekt, og der tjekkes om transaktionen er gennemført. Er den ikke det, må det skyldes at balancen på spillerens konto er blevet negativ, så spillet afsluttes og den aktuelle spiller erklæres som taber af spillet. Bliver transaktionen gennemført fortættes spillet, og der udskrives en status. Herefter tjekkes om spilleren har opnået det ønskede antal point for at vinde – har han det, afsluttes spillet og spilleren erklæres som vinder. Har han ikke det, fortsættes spillet, og der tjekkes om det felt, spilleren er landet på, giver en ekstra tur. Gør det det, køres en ny tur med den samme spiller, gør det ikke det, skiftes til den næste spiller i rækkefølgen, før der køres en ny tur.
\\

For hver af alle disse handlinger (afslut spillet, erklær en spiller som taber, erklær en spiller som vinder, udskriv status, skift til næste spiller i rækkefølgen), findes en hjælpemetode, som udfører de relevante operationer for en handling.
\\

Når spillet afsluttes udføres en oprydning af de benyttede komponenter.
\begin{figure}[!ht]
\centering
\includegraphics[scale=0.4]{Game-illustration3.jpg}
\caption[<Text for the list of figures>]{Metode til oprydning}
\label{fig:figure 2} 
\end{figure}
GUI’en lukkes, scanneren, som benyttes til at hente input fra konsollen, lukkes, og til sidst afsluttes programmet.
\\

Når en spiller erklæres som taber, udskrives navnet og scoren for taberen, og der ventes på konsolinput, for at sikre at spillerne når at se beskeden, inden spillet lukker. Når der gives et input, lukkes spillet med cleanUp metoden ovenfor.
\begin{figure}[!ht]
\centering
\includegraphics[scale=0.4]{Game-illustration4.jpg}
\caption[<Text for the list of figures>]{Metode for tabt spil}
\label{fig:figure 2} 
\end{figure}
Præcist det samme gør sig gældende når en spiller vinder, bortset selvfølgelig fra at teksten som udskrives er anderledes.
\\

Når der udskrives status, foregår det ligeledes med en hjælpemetode. Herfra opdateres både GUI og TUI.
\begin{figure}[!ht]
\centering
\includegraphics[scale=0.4]{Game-illustration5.jpg}
\caption[<Text for the list of figures>]{Metode til opdatering af TUI og GUI}
\label{fig:figure 2} 
\end{figure}
Der udskrives en status til TUI’en terningernes værdi, samt score for alle spillere. Derudover indstilles terningerne på GUI’en, ligesom spillerens score opdateres og den aktuelle spillers placering på spillepladen ændres.
\\

På samme måde er der også en metode til at skifte til den næste bruger i rækkefølgen.
\begin{figure}[!ht]
\centering
\includegraphics[scale=0.4]{Game-illustration6.jpg}
\caption[<Text for the list of figures>]{Metode til næste spiller}
\label{fig:figure 2} 
\end{figure}
Igen er det her værd at bemærke, at metoden også vil virke med mere end 2 spillere. Metoden tager nummeret på den nuværende spiller som input, og lægger en til – med mindre den spiller der i så fald skulle returneres ville have et højere nummer, end den konstant der er angivet for antallet af spillere – i så fald returneres blot 0 (dvs. der startes forfra i rækkefølgen).
\section{Test}
\subsection{Test af Account-klassen}
Der skal laves en test, der sandsynliggør at balancen i Account aldrig kan blive negativ, uanset hvilken værdi set og add metoderne kaldes med. Dvs. en blackbox test af Account-klassen, hvor vi ikke bekymrer os om hvad der sker inde i Account, men blot tester hvad der kommer ud, ift. hvad vi har puttet ind.
\\

Til formålet oprettes en testklasse, ”AccountTesterController”, som tester grænseværdier, store værdier og mere gennemsnitlige værdier. Klassen opretter en ny Account, og kalder derefter ”setAccountValue” og ”addToAccount” med de forskellige værdier. Der udskrives så en kort beskrivelse af hver test, samt resultatet af testen. Herunder outputtet fra metoden.
\begin{figure}[!ht]
\centering
\includegraphics[scale=0.4]{test-illustrationer1.jpg}
\caption[<Text for the list of figures>]{Resultatet af testkørsel}
\label{fig:figure 2} 
\end{figure}
Dvs. hvis kontoen sættes til 1, bliver transaktionen gennemført (metoden returnerer ”true”). Hvis kontoen sættes til 0, bliver transaktionen ligeledes gennemført osv.
\\

Det interessante her er især grænseværdierne, dvs. netop 1, 0 og -1. Testen viser, at klassen opfører sig som vi ville forvente – en positiv værdi giver selvfølgelig true, men mere interessant, 0 giver også true. Ligeledes som forventet, returneres false, hvis balancen sættes til -1.
\\

Når grænseværdierne virker som forventet, vil alle værdier typisk gøre det, men for en sikkerheds skyld testes også lige med nogle meget store værdier, og med nogle helt almindelige værdier. Disse test giver også det forventede resultat.
\\

Herefter udføres en test hvor balancen først sættes til noget kendt, og derefter opdateres med ”addToAccount”. Ved den første test sættes balancen til 1000, hvorefter tilføjes -1000 – således må det forventes at den resulterende balance bliver 0. Dermed vil vi forvente at tranaktionen bliver fuldført, idet 0 ikke er negativt, og dermed er en godkendt værdi. Vi ser, at også denne test giver det forventede resultat.
\\

Sæt til 1000 og tilføj -1001 må give -1, og dermed false – OK.
Sæt til 1001 og tilføj -1000 må give 1, og dermed true – OK.
\\

De mere almindelige værdier giver ligeledes det forventede resultat. Således må det være fair at sige, at det er sandsynligt at en transaktion kun bliver gennemført, hvis ikke den resulterer i en negativ balance.
\\

\subsection{Test og fejlfinding generelt}
Foruden klassen til blackbox-test af Account-klassen, er der udviklet to test-klasser, som kan bruges til fejlfinding og test i programmet. Tanken er, at en eller begge test-klasser kan erstatte de ”rigtige” klasser, og på den måde give en udvikler mulighed for at teste scenarier, som er svære og/eller tidskrævende at opnå ved almindeligt spil.
\\

Der er dels tale om en GameBoard test-klasse, hvor værdierne for felterne alle er negative, således at sikringen af Account-klassen hurtigt kan testes i praksis. Derudover er der tale om en DieCup test-klasse, hvor en udvikler selv har mulighed for at sætte en fast værdi for hvad terningerne skal slå. Således kan der opnås at lande på f.eks. felt nr. 3 (som giver -200) mange gange i træk, og derved teste sikringen af Account-klassen, eller der kan landes på felt 12 (som giver +650) flere gange i træk, og derved teste de handlinger som udføres når en spiller vinder
\section{Design sekvens-diagram}
\todo[inline]{Chris: tegn diagram og forklar}
\begin{figure}[!ht]
\centering
\includegraphics[width=1\textwidth]{DSD.pdf}
\caption[<Text for the list of figures>]{Design Sekvens Diagram}
\label{fig:figure2}
\end{figure}
\newpage
Vi har, som ved første spil,  lavet et design sekvens-diagram, som dokumenterer, hvordan koden bliver udført sekventielt, når man spiller det nye feltspil. Hvis man sammenligner vores sekvensdiagram med vores kode afsnit, vil man se at \textit{Game} controlleren har det primære ansvar for at fordele opgaverne rundt til de ande dele af programmet. Her kan vi bare se det ud fra et tidsmæssigt perspektiv istedet. Hvordan de forskellige metodekald virker, kan man nærstudere i vores afsnit om selve koden.
\\

Desværre er diagrammet fyldt med forkert syntaks og variabler, der er brugt som metodekald. Da dette er noget af det sidste vi har lavet i vores dokumentation, har tidsgrænsen bevirket, vi ikke har kunnet afhjælpe disse fejl inden for den ønskede tidsramme.
\\
Vi vil dog anbefale, at man får dette rettet hurtigst muligt, hvis der er planer om eventuelle opdateringer til spillet, eller genbrug af dele af systemet.
\section{Design-klassediagram}
\todo[inline]{Thomas A: lav diagram og forklar}
\begin{figure}[ht]
\centering
\includegraphics[scale=0.4]{DesignClass.jpg}
\caption[<Text for the list of figures>]{Design-Klassediagram}
\label{fig:figure2}
\end{figure}
Vi har også valgt at dokumentere vores program med et klassediagram. Dette giver et godt overblik over programmets opbygning, og om vi har overholdt tankegangen for \textbf{GRASP}. Denne tankegang vil blive beskrevet i næste afsnit.
\section{GRASP (General Responsibilty Assignment Software Patterns)}
\todo[inline]{Thomas A: Forklar hvad vi har brugt}
Vi vil i dette afsnit beskrive de forskellige GRASP patterns og give eksempler på, hvordan vi har brugt det til implementering af vores program \cite{Larman}.
\subsection{Controller}
En controller er et vækrtøj (eller en funktion) i et program der angiver hvad der skal ske i programmet i den rigtige rækkefølge. Man kan næsten sige at det er direktøren i et firma der søger for alt kører som det skal.
\subsubsection*{Problem}
Hvilket objet under UI laget kontrollerer system operationer
\subsubsection*{System operationer}
System operationer støder vi første gang på under analysen af \textbf{SSD} \textit{(System Sekvens Diagram)}. Dette er de vigtige hændelser i vores system.
For eksempel, når en spiller i vores spil trykker enter for at rulle med terningerne. Her starter han en system hændelse, der giver et terningeslag.
\\
En \textbf{Controller} er det første object under vores \textbf{UI} \textit{User Interface}, der har ansvar for at modtage eller løse system operations beskeder.
\subsubsection*{Generel Løsning}
Tildel ansvaret til en klasse, som benytter en af følgende to valgmuligheder.
\begin{enumerate}
\item Klassen repræsenterer hovedsystemet med en slags "rod object". En enhed, sofwaren kører inden i systemet, eller et decideret subsystem - Dette er variationer af en \textit{Facade Controller}.
\item Klassen repræsenterer et use case scenarie hvor denne system handling ofte fremkommer. Denne vil tit være kaldet <UseCaseName>Handler, <UseCaseName>Coordinator eller <UseCaseName>Session.
\end{enumerate}
\subsubsection*{Vores Løsning}
Vi har lavet en \textbf{Controller} kaldet \textit{GameController} til at styre vores sekvenser i spillet efter princippet med en \textit{Facade Controller}. 
\\
Det giver meget god mening at have en klasse, der uddelegerer ansvar. Dette gør at \textit{GameController} kun skal kontrollere og kordinere opgaver til andre objekter. Samtidig skal \textit{GameController} ikke udføre meget arbejde selv. Dette opfylder den guideline, der findes i \textit{Larman}
\subsection{Creater}
<<<<<<< HEAD
En creator er bogstaveligt tatlt det. Den skaber objekter af en klasse. Det er en funktion man bruger meget til at gøre programmering mere simpel og effektiv. Nedenfor kan der ses generelle situationer hvori, dette bruges:
\begin{enumerate}
\item Instancer af B indeholder aggregerede instancer af A
\item Instancer af B iagtager instancer af A
\item Instancer af B ligger tæt op af instancer af A
\item Instancer af B indeholder den igang-sættende information for instancer af A og senere krearer det
\end{enumerate}
Et godt eksempel fra vores egen kode kunne være vores Ownable klasse. Den indeholder instancer af Field. Ownable "reagere" kun på hvorvidt instancerne i Field bliver aktiveret. 
\subsubsection*{Problem}
Oprettelse af objekter, er en af de mest almindelige aktiviteter i et objektorienteret system.
\\
\subsubsection*{Problem}
Oprettelse af objekter, er en af de mest almindelige aktiviteter i et objektorienteret system.
\\
Et generelt princip til oprettelses ansvar er meget brugbart. Hvis ansvarsfordelingen bliver fordelt godt, kan man opnå \textbf{lav kobling} \textit{(Andet GRAS Pattern, der beskrives senere)}, større klarhed, indkapsling (den internne representation af et objekt, der er gemt fra kig udenfor objektets definition.) og genanvendelighed.
\subsubsection*{Generel Løsning}
Giv klasse B ansvar for at oprette en instans af klassen A, hvis et af disse udtryk er sande (Helst flere af udtrykkenne).
\begin{itemize}
\item B indeholder, eller komposit agrigerer A
\item B bruger A tæt.
\item Bhar de initialiserende data til A, som bliver sendt til A, når denne er oprettet.
\\
Dermed er B \textbf{Expert} \textit{(Andet GRAS Pattern, der beskrives senere)} for at oprette A.
\end{itemize}
\subsubsection*{Vores Løsning}
Denne tankegang har vi brugt i forbindelse med vores domænemodel. I den kan vi udlede at vores \textit{Game} bruger et \textit{GameBoard}, som indeholder flere \textit{Fields}. Vi kan dermed se at \textit{Game} er en god kandidat til at bære ansvaret for at oprette \textit{GameBoard}. \textit{GameBoard} er også oplagt til at oprette \textit{Field} objekter. Samme historie gentages i resten af modellen, så godt som muligt efter denne tankegang.
=======
I forbindelse med vores domænemodel kan vi se at vores \textit{Game} bruger et \textit{GameBoard}, som indeholder flere \textit{Fields}. Vi kan dermed se at \textit{Game} er en god kandidat til at bære ansvaret for at oprette \textit{GameBoard}. \textit{GameBoard} er også oplagt til at oprette \textit{Field} objekter. Samme historie gentages i resten af modellen, så godt som muligt efter denne
>>>>>>> a0925f260e9b30580f63c58398b2556d7e2d6d6b
\subsection{Expert}
I vores kode har der været brug for en expert til at holde styr på vores felter på spillepladen. Derfor har vi lagret alle informationer, såsom hvad en grund koster at købe eller hvad den koster at lande på når en anden ejer den, i \textit{gameBoard} klassen. Det gør samtidigt at vi hurtigt kan komme til informationerne fra andre klasser da vi kun skal gå et sted hen for at hente informationerne. 
Man kan bruge et eksempel fra den virkelige verden. Forestil dig at du skal slå 20 dyr op. Hvis du skal slå op i en bog for hvert dyr kan det tage sin tid, i forhold til hvis du kun skal have fat i en bog.
\\
\subsection{High Cohesion (Høj binding)}
Høj binding er noget man altid stræber efter i et system. Det har vi også gjort som man kan se på vores \textit{Account} og \textit{Player} klasser. De
\subsection{Indirection}
Beskrivelse af Indirection
\subsubsection*{Problem}
Hvor skal man tildele et ansvarsområder, for at undgå direkte kobling mellem to eller flere ting?
\\
Hvordan afkobler man objekter, så man opnår lav kobling, og genanvendelsesmuligheder forbliver høje.
\subsubsection*{Generel Løsning}
Giv ansvaret til et mellemliggende objekt til at kommunikere mellem andre komponenter, så de ikke er direkte sammenkoblet.
\\
Det mellemliggende objekt laver en \textit{inderection} mellem de andre komponenter.
\subsubsection*{Vores Løsning}
Vi har forsøgt at bruge \textit{Inderection}, ved at lave en \textit{Graphic} klasse, der holder styr på alt, der foregår i forhold til at kalde vores eksterne GUI. Dette gjorde vi for, at det ville være nemmere at lave spillet om, hvis der skulle ske modifikationer i GUI biblioteket. Dette skete rent faktisk, da vi var midt i projektet. Vi fik besked om at GUI'en, nu var blevet ændret. Selvom vi syntes det var lidt irriterende, kunne vi hurtigt rette det til, da vi kun skulle rette i den ene klasse.
\subsection{Low Coupling (Lav kobling)}
Beskrivelse af Low Coupling
\subsubsection*{Problem}
Hvordan opnår man lav afhængighed. lav forandrings indvirkning og forhøjet genanvendelse.
\\
\textbf{Coubling} er et mål for hvor stærkt et element er koblet til, har kendskab til eller er afhængig af andre elementer.
\\
En klasse med høj kobling afhænger af mange klasser. Sådanne klasser kan være uønskede. De kan lide af følgende problemer.
\begin{itemize}
\item Tvunge lokale forandringer på grund af forandringer i sammenhængende klasser.
\item Sværere at forstå i isolerede tilfælde.
\item Sværere at genanvende, da det vil kræve tilhørende tilstedeværelse af afhængige klasser.
\end{itemize}
\subsubsection*{Generel Løsning}
Tildel ansvar, så koblingen forbliver lav. Brug dette princip til at evaluere alternativer.
\subsubsection*{Vores Løsning}
Dette er forsøgt implementeret i for eksempel vores \textit{Game}, der istedet for at kalde til  \textit{Player} for bagefter at kalde \textit{Account}, og dermed skabe høj kobling, har vi valgt at kalde \textit{Account} igennem \textit{Player} klassen. Dette sikrer en lav kobling, men samtidig også høj binding, da man sjældent kan bruge principperne alene.
\subsection{Polymorphism}
Beskrivelse af Polymorphism
\subsubsection*{Problem}
Hvordan opretter man alternativer baseret på typer eller software, der kan kobles direkte på de allerede eksisterende komponenter.
\textit{Alternativer baseret på typer} - Hvis et program er designet med et if-else eller switch statement, og en ny variation opstår, kan det ofte betyde ændringer mange steder i koden. Denne fremgangsmåde gør det besværligt at udvide et program på en nem måde. Dette er fordi, ændringerne skal foretages flere forskellige steder, hvor denne betingelses logik er implementeret.
\subsubsection*{Generel Løsning}
Når familiære alternativer eller opførsel varierer efter type (klasse), gives ansvaret for opførslen til typerne, hvori opførslen varierer. Dette gøres ved at bruge polymorfiske operatgioner.
\subsubsection*{Vores Løsning}
Vi har ikke benyttet polymorphi i vores spil. Vi har dog tænkt på, at vores felter måske i fremtiden, vil komme til at være af forskellige typer. Dermed kan det blive aktuelt at benytte mønstret for polymorfi ved en eventuel opdatering af spillet.
\subsection{Protected Variations}
Beskrivelse af Protected Variations
\subsubsection*{Problem}
Hvordan tildeler man ansvar til objekter, subsystemer eller systemer, så variationer og ustabilitet i disse elementer ikke får en uønsket effekt på andre elementer.
\subsubsection*{Generel Løsning}
Identificer punkter med uønsket variation eller ustabilitet. Lav en stabil grænseflade om dem. Dette bruges tit i forbindelse med polymorfi.
\subsubsection*{Vores Løsning}
I vores spil har det ikke været nødvendigt at bruge dette mønster.
\subsection{Pure Fabrication}
Beskrivelse af Pure Fabrication
\subsubsection*{Problem}
Objekt orienterede designs er nogle gange karakteriserede ved at tage udgangspunkt i problemdomæner fra den virkelige verden, for at lette forståelsen. For eksempel \textit{Gameboard} og \textit{Field} klasserne. Nogle gange kan der der opstå situationer, hvor det giver problemer kun at tildele ansvar til domæne lags klasser. Det kan give dårlig binding, kobling eller lav genanvedelsesmulighed.
\subsubsection*{Generel Løsning}
Giv et sæt ansvarsopgaver med høj sammenhæng til en kunstig eller belejelig klasse, der ikke er repræsenteret i domænelaget. Denne opfundne klasse skal supportere høj binding, lav kobling og være nem at genbruge.
\subsubsection*{Vores Løsning}
Vi har opfundet klasser til at håndtere det grafiske i spillet med klassen \textit{Graphic}, og den tekstbaserede del kørers i klassen \textit{TUI}. Disse klasser opfylder ovenstående krav.
\section{FURPS+}
\todo[inline]{Thomas M: forklar furps, og hvordan vi har brugt det.}
I \textbf{UP} bruger man \textbf{FURPS+}, der er udviklet af Hewlett-Packard til at kategoriserer kravene til ens system. "\textbf{+}" i \textbf{FURPS+} kom i følge \cite{WikiFURPS} til senere, efter HP ønskede at dække flere kategorier med denne model.
\subsection{Functional}
Hvilke funktioner skal systemt have. Hvad skal det kunne. Sikkerhed.
\subsection{Usability}
human factors, help, documentation.
\subsection{Reliability}
requency of failure, recoverability, predictability.
\subsection{Performance}
response times, throughput, accuracy, availability, resource
\subsection{Supportability}
daptability, maintainability, internationalization, configurability.
\subsubsection*{Her kommer de underfaktorer, som +'et repræsenterer}
\subsection{Implementation}
resource limitations, languages and tools, hardware,
\subsection{Interface}
constraints imposed by interfacing with external systems.
\subsection{Operations}
system management in its operational setting
\subsection{Packaging}
How to deliver the system? Physical box or file...
How many installations are there?
\subsection{Legal}
licensing and so forth.
\section{Kildeliste}
Her vil vi oplyse om hvilke kilder, der er brugt til rapporten.
Vi vil både oplyse om hvilke bøger, hjemmesider og software vi har brugt.
\subsection{Bøger/rapporter}
\printbibliography %Print the bibliography set under \bibliography.
\begin{itemize}
\item Applying UML and patterns - An introduction to Object-Oriented Analysis and Design and Iterative Development - Craig Larman, Third Edition
\item Java Software Solutions -  Foundations of Program Design - Lewis and Loftus, Seventh Edition
\item 19\_del1 - Ahlgreen, Johansen og Mortensen
\end{itemize}
\subsection{Hjemmesider}
\subsection{Programmer}
\begin{itemize}
\item Eclipse - kepler
\item UMLet
\item Google Docs
\item TexMaker
\end{itemize}
\newpage
\section{Bilag}
\subsection{Kode}
Her vil hele vores kode til programmet være repræsenteret som bilag.
\subsubsection{TUI - Boundary}
\lstinputlisting[language = Java, tabsize = 2,stringstyle=\color{myblue},commentstyle=\color{mygreen},showstringspaces = false, breaklines=true, numbers = left,keywordstyle = \bfseries\color{mymauve}]{TUI.java}
\subsubsection{Graphic - Boundary}
\lstinputlisting[language = Java, tabsize = 2,stringstyle=\color{myblue},commentstyle=\color{mygreen},showstringspaces = false, breaklines=true, numbers = left,keywordstyle = \bfseries\color{mymauve}]{Graphic.java}
\subsubsection{Main - Controller}
\lstinputlisting[language = Java, tabsize = 2,stringstyle=\color{myblue},commentstyle=\color{mygreen},showstringspaces = false, breaklines=true ,numbers = left,keywordstyle = \bfseries\color{mymauve}]{Main.java}
\subsubsection{Game - Controller}
\lstinputlisting[language = Java, tabsize = 2,stringstyle=\color{myblue},commentstyle=\color{mygreen},showstringspaces = false, breaklines=true, numbers = left,keywordstyle = \bfseries\color{mymauve}]{Game.java}
\subsubsection{Player - Entity}
\lstinputlisting[language = Java, tabsize = 2,stringstyle=\color{myblue},commentstyle=\color{mygreen},showstringspaces = false, breaklines=true, numbers = left,keywordstyle = \bfseries\color{mymauve}]{Player.java}
\subsubsection{Account - Entity}
\lstinputlisting[language = Java, tabsize = 2,stringstyle=\color{myblue},commentstyle=\color{mygreen},showstringspaces = false,breaklines=true, numbers = left,keywordstyle = \bfseries\color{mymauve}]{Account.java}
\subsubsection{Gameboard - Entity}
\lstinputlisting[language = Java, tabsize = 2,stringstyle=\color{myblue},commentstyle=\color{mygreen},showstringspaces = false,breaklines=true, numbers = left,keywordstyle = \bfseries\color{mymauve}]{Gameboard.java}
\subsubsection{Field - Entity}
\lstinputlisting[language = Java, tabsize = 2,stringstyle=\color{myblue},commentstyle=\color{mygreen},showstringspaces = false,breaklines=true, numbers = left,keywordstyle = \bfseries\color{mymauve}]{Field.java}
\subsubsection{DieCup - Entity}
\lstinputlisting[language = Java, tabsize = 2,stringstyle=\color{myblue},commentstyle=\color{mygreen},showstringspaces = false, breaklines=true, numbers = left,keywordstyle = \bfseries\color{mymauve}]{DieCup.java}
\subsubsection{Die - Entity}
\lstinputlisting[language = Java, tabsize = 2,stringstyle=\color{myblue},commentstyle=\color{mygreen},showstringspaces = false,breaklines=true, numbers = left,keywordstyle = \bfseries\color{mymauve}]{Die.java}
\subsubsection{AccountTesterController - TestTools}
\lstinputlisting[language = Java, tabsize = 2,stringstyle=\color{myblue},commentstyle=\color{mygreen},showstringspaces = false, breaklines=true, numbers = left,keywordstyle = \bfseries\color{mymauve}]{AccountTesterController.java}
\subsubsection{DieCupTestEntity - TestTools}
\lstinputlisting[language = Java, tabsize = 2,stringstyle=\color{myblue},commentstyle=\color{mygreen},showstringspaces = false, breaklines=true, numbers = left,keywordstyle = \bfseries\color{mymauve}]{DieCupTestEntity.java}
\subsubsection{GameBoardTestEntity - TestTools}
\lstinputlisting[language = Java, tabsize = 2,stringstyle=\color{myblue},commentstyle=\color{mygreen},showstringspaces = false, breaklines=true, numbers = left,keywordstyle = \bfseries\color{mymauve}]{GameBoardTestEntity.java}
\end{document}