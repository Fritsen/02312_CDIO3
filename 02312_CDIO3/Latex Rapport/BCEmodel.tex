\section{BCE model}
\todo[inline]{Thomas M: indsæt model og forklar}
\begin{figure}[ht]
\centering
\includegraphics[width=1\textwidth]{BCEModel.pdf}
\caption[<Text for the list of figures>]{BCE Model}
\label{fig:bcemodel}
\end{figure}
I denne CDIO opgave, har vi igen benyttet os af en BCE model til at skabe oveblik over vores kendskab i koden.
\\
Vi har denne gang ændret 
\\
Det vil sige at vi i denne model kun har tilføjet vores nye entitetsklasser. Det handler om entiteterne \textit{Account}, \textit{Field}, og \textit{GameBoard}. Vi har også lavet en ny boundary vi kalder \textit{Graphic}, som bearbejder beskeder til vores importerede \textit{GUI}.
\\
Som det fremgår af diagrammet, styrer controlleren \textit{Game} stadig spillet. Det vil sige at vores controller tildeler ansvaret ud til de nære klasser.
\\
\textit{Account} er en pengebeholdning der styres af \textit{Player} klassen. Vores controller har derfor kun brug for direkte kendskab til \textit{Player}.
\\\\
\textit{Field} holder styr på felterne i spillet, og vores \textit{Gameboard} holder styr på felterne. Vores \textit{Game} controller har dermed kendskab til felterne i spillet gennem vores \textit{Gameboard}.
\\


Ud fra disse informationer kan controlleren så sende besked til vores boundary's \textit{TUI} og \textit{Graphic}, som viser brugeren konsekvenserne i spillet grafisk og i tekst. Dette er hvad \vref{fig:bcemodel} beskriver