\section{GRASP (General Responsibilty Assignment Software Patterns)}
\todo[inline]{Thomas A: Forklar hvad vi har brugt}
Vi vil i dette afsnit beskrive de forskellige GRASP patterns og give eksempler på, hvordan vi har brugt det til implementering af vores program \cite{Larman}.
\subsection{Controller}
En controller er et vækrtøj (eller en funktion) i et program der angiver hvad der skal ske i programmet i den rigtige rækkefølge. Man kan næsten sige at det er direktøren i et firma der søger for alt kører som det skal.
\subsubsection*{Problem}
Hvilket objet under UI laget kontrollerer system operationer
\subsubsection*{System operationer}
System operationer støder vi første gang på under analysen af \textbf{SSD} \textit{(System Sekvens Diagram)}. Dette er de vigtige hændelser i vores system.
For eksempel, når en spiller i vores spil trykker enter for at rulle med terningerne. Her starter han en system hændelse, der giver et terningeslag.
\\
En \textbf{Controller} er det første object under vores \textbf{UI} \textit{User Interface}, der har ansvar for at modtage eller løse system operations beskeder.
\subsubsection*{Generel Løsning}
Tildel ansvaret til en klasse, som benytter en af følgende to valgmuligheder.
\begin{enumerate}
\item Klassen repræsenterer hovedsystemet med en slags "rod object". En enhed, sofwaren kører inden i systemet, eller et decideret subsystem - Dette er variationer af en \textit{Facade Controller}.
\item Klassen repræsenterer et use case scenarie hvor denne system handling ofte fremkommer. Denne vil tit være kaldet <UseCaseName>Handler, <UseCaseName>Coordinator eller <UseCaseName>Session.
\end{enumerate}
\subsubsection*{Vores Løsning}
Vi har lavet en \textbf{Controller} kaldet \textit{GameController} til at styre vores sekvenser i spillet efter princippet med en \textit{Facade Controller}. 
\\
Det giver meget god mening at have en klasse, der uddelegerer ansvar. Dette gør at \textit{GameController} kun skal kontrollere og kordinere opgaver til andre objekter. Samtidig skal \textit{GameController} ikke udføre meget arbejde selv. Dette opfylder den guideline, der findes i \textit{Larman}
\subsection{Creater}
En creator er bogstaveligt tatlt det. Den skaber objekter af en klasse. Det er en funktion man bruger meget til at gøre programmering mere simpel og effektiv. Nedenfor kan der ses generelle situationer hvori, dette bruges:
\begin{enumerate}
\item Instancer af B indeholder aggregerede instancer af A
\item Instancer af B iagtager instancer af A
\item Instancer af B ligger tæt op af instancer af A
\item Instancer af B indeholder den igang-sættende information for instancer af A og senere krearer det
\end{enumerate}
Et godt eksempel fra vores egen kode kunne være vores Ownable klasse. Den indeholder instancer af Field. Ownable "reagere" kun på hvorvidt instancerne i Field bliver aktiveret. 
\subsubsection*{Problem}
Oprettelse af objekter, er en af de mest almindelige aktiviteter i et objektorienteret system.
\\
\subsubsection*{Problem}
Oprettelse af objekter, er en af de mest almindelige aktiviteter i et objektorienteret system.
\\
Et generelt princip til oprettelses ansvar er meget brugbart. Hvis ansvarsfordelingen bliver fordelt godt, kan man opnå \textbf{lav kobling} \textit{(Andet GRAS Pattern, der beskrives senere)}, større klarhed, indkapsling (den internne representation af et objekt, der er gemt fra kig udenfor objektets definition.) og genanvendelighed.
\subsubsection*{Generel Løsning}
Giv klasse B ansvar for at oprette en instans af klassen A, hvis et af disse udtryk er sande (Helst flere af udtrykkenne).
\begin{itemize}
\item B indeholder, eller komposit agrigerer A
\item B bruger A tæt.
\item Bhar de initialiserende data til A, som bliver sendt til A, når denne er oprettet.
\\
Dermed er B \textbf{Expert} \textit{(Andet GRAS Pattern, der beskrives senere)} for at oprette A.
\end{itemize}
\subsubsection*{Vores Løsning}
Denne tankegang har vi brugt i forbindelse med vores domænemodel. I den kan vi udlede at vores \textit{Game} bruger et \textit{GameBoard}, som indeholder flere \textit{Fields}. Vi kan dermed se at \textit{Game} er en god kandidat til at bære ansvaret for at oprette \textit{GameBoard}. \textit{GameBoard} er også oplagt til at oprette \textit{Field} objekter. Samme historie gentages i resten af modellen, så godt som muligt efter denne tankegang.
\subsection{Expert}
Beskrivelse af en Expert
\subsubsection*{Problem}
Hvad er den gennerelle fremgangsmåde for at tildele ansvar til objekter.
\subsubsection*{Generel Løsning}
Tildel ansvar til informations eksperten. Dette er klassen, der har de nødvendige informationer til at fuldfylde det ansvar.
\subsubsection*{Vores Løsning}
Hvis vi kigger på eksempelvis vores \textit{DieCup}, har den ansvaret for at kende \textit{Die} terningeslag. Dermed er \textit{Diecup} expert i at få terningeslag.
\subsection{High Cohesion (Høj binding)}
Beskrivelse af High Cohesion
\subsubsection*{Problem}
Hvordan holder man objekter fokuserede, forståelige og håndterbare og som en sideeffekt benytter lav kobling \textit{(Andet GRAS Pattern, der beskrives senere)}
\subsubsection*{Generel Løsning}
Tildel ansvar, så bindingen forbliver høj. Brug dette til at vurdere alternativer.
\\
En klasse med lav binding laver mange urelaterede ting, eller har for mange opgaver.
\\
Sådanne klasser ønskes ikke fordi de kan være:
\begin{itemize}
\item Svær at forstå
\item Svær at genanvende
\item Konstant udsat for forandringer
\end{itemize}
Klasser med lav binding har tit for meget ansvar, som kunne uddelegeres til andre objekter.
\subsubsection*{Vores Løsning}
Et godt eksempel i vores program, er at vores \textit{Game}klasse,  holder styr på spillerne i \textit{Player} klassen. Istedet for at \textit{Game} også holder styr på spillerens pengebeholdning, har vi givet ansvaret til \textit{Player}, som holder styr på \textit{Account}.
\subsection{Indirection}
Beskrivelse af Indirection
\subsubsection*{Problem}
Hvor skal man tildele et ansvarsområder, for at undgå direkte kobling mellem to eller flere ting?
\\
Hvordan afkobler man objekter, så man opnår lav kobling, og genanvendelsesmuligheder forbliver høje.
\subsubsection*{Generel Løsning}
Giv ansvaret til et mellemliggende objekt til at kommunikere mellem andre komponenter, så de ikke er direkte sammenkoblet.
\\
Det mellemliggende objekt laver en \textit{inderection} mellem de andre komponenter.
\subsubsection*{Vores Løsning}
Vi har forsøgt at bruge \textit{Inderection}, ved at lave en \textit{Graphic} klasse, der holder styr på alt, der foregår i forhold til at kalde vores eksterne GUI. Dette gjorde vi for, at det ville være nemmere at lave spillet om, hvis der skulle ske modifikationer i GUI biblioteket. Dette skete rent faktisk, da vi var midt i projektet. Vi fik besked om at GUI'en, nu var blevet ændret. Selvom vi syntes det var lidt irriterende, kunne vi hurtigt rette det til, da vi kun skulle rette i den ene klasse.
\subsection{Low Coupling (Lav kobling)}
Beskrivelse af Low Coupling
\subsubsection*{Problem}
Hvordan opnår man lav afhængighed. lav forandrings indvirkning og forhøjet genanvendelse.
\\
\textbf{Coubling} er et mål for hvor stærkt et element er koblet til, har kendskab til eller er afhængig af andre elementer.
\\
En klasse med høj kobling afhænger af mange klasser. Sådanne klasser kan være uønskede. De kan lide af følgende problemer.
\begin{itemize}
\item Tvunge lokale forandringer på grund af forandringer i sammenhængende klasser.
\item Sværere at forstå i isolerede tilfælde.
\item Sværere at genanvende, da det vil kræve tilhørende tilstedeværelse af afhængige klasser.
\end{itemize}
\subsubsection*{Generel Løsning}
Tildel ansvar, så koblingen forbliver lav. Brug dette princip til at evaluere alternativer.
\subsubsection*{Vores Løsning}
Dette er forsøgt implementeret i for eksempel vores \textit{Game}, der istedet for at kalde til  \textit{Player} for bagefter at kalde \textit{Account}, og dermed skabe høj kobling, har vi valgt at kalde \textit{Account} igennem \textit{Player} klassen. Dette sikrer en lav kobling, men samtidig også høj binding, da man sjældent kan bruge principperne alene.
\subsection{Polymorphism}
Beskrivelse af Polymorphism
\subsubsection*{Problem}
Hvordan opretter man alternativer baseret på typer eller software, der kan kobles direkte på de allerede eksisterende komponenter.
\textit{Alternativer baseret på typer} - Hvis et program er designet med et if-else eller switch statement, og en ny variation opstår, kan det ofte betyde ændringer mange steder i koden. Denne fremgangsmåde gør det besværligt at udvide et program på en nem måde. Dette er fordi, ændringerne skal foretages flere forskellige steder, hvor denne betingelses logik er implementeret.
\subsubsection*{Generel Løsning}
Når familiære alternativer eller opførsel varierer efter type (klasse), gives ansvaret for opførslen til typerne, hvori opførslen varierer. Dette gøres ved at bruge polymorfiske operatgioner.
\subsubsection*{Vores Løsning}
Vi har ikke benyttet polymorphi i vores spil. Vi har dog tænkt på, at vores felter måske i fremtiden, vil komme til at være af forskellige typer. Dermed kan det blive aktuelt at benytte mønstret for polymorfi ved en eventuel opdatering af spillet.
\subsection{Protected Variations}
Beskrivelse af Protected Variations
\subsubsection*{Problem}
Hvordan tildeler man ansvar til objekter, subsystemer eller systemer, så variationer og ustabilitet i disse elementer ikke får en uønsket effekt på andre elementer.
\subsubsection*{Generel Løsning}
Identificer punkter med uønsket variation eller ustabilitet. Lav en stabil grænseflade om dem. Dette bruges tit i forbindelse med polymorfi.
\subsubsection*{Vores Løsning}
I vores spil har det ikke været nødvendigt at bruge dette mønster.
\subsection{Pure Fabrication}
Beskrivelse af Pure Fabrication
\subsubsection*{Problem}
Objekt orienterede designs er nogle gange karakteriserede ved at tage udgangspunkt i problemdomæner fra den virkelige verden, for at lette forståelsen. For eksempel \textit{Gameboard} og \textit{Field} klasserne. Nogle gange kan der der opstå situationer, hvor det giver problemer kun at tildele ansvar til domæne lags klasser. Det kan give dårlig binding, kobling eller lav genanvedelsesmulighed.
\subsubsection*{Generel Løsning}
Giv et sæt ansvarsopgaver med høj sammenhæng til en kunstig eller belejelig klasse, der ikke er repræsenteret i domænelaget. Denne opfundne klasse skal supportere høj binding, lav kobling og være nem at genbruge.
\subsubsection*{Vores Løsning}
Vi har opfundet klasser til at håndtere det grafiske i spillet med klassen \textit{Graphic}, og den tekstbaserede del kørers i klassen \textit{TUI}. Disse klasser opfylder ovenstående krav.