\section{FURPS+}
\todo[inline]{Thomas M: forklar furps, og hvordan vi har brugt det.}
I \textbf{UP} bruger man \textbf{FURPS+}, der er udviklet af Hewlett-Packard til at kategoriserer kravene til ens system. "\textbf{+}" i \textbf{FURPS+} kom i følge \cite{WikiFURPS} til senere, efter HP ønskede at dække flere kategorier med denne model.
\subsection{Functional}
Hvilke funktioner skal systemt have. Hvad skal det kunne. Sikkerhed.
\subsection{Usability}
human factors, help, documentation.
\subsection{Reliability}
requency of failure, recoverability, predictability.
\subsection{Performance}
response times, throughput, accuracy, availability, resource
\subsection{Supportability}
daptability, maintainability, internationalization, configurability.
\subsubsection*{Her kommer de underfaktorer, som +'et repræsenterer}
\subsection{Implementation}
resource limitations, languages and tools, hardware,
\subsection{Interface}
constraints imposed by interfacing with external systems.
\subsection{Operations}
system management in its operational setting
\subsection{Packaging}
How to deliver the system? Physical box or file...
How many installations are there?
\subsection{Legal}
licensing and so forth.